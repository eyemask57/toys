\hypersetup{
	luatex,
	pdfencoding=auto,
	unicode,
	hidelinks,
	pdfusetitle,
	bookmarks=true,
	bookmarksdepth=3,
	bookmarksnumbered=true,
	colorlinks=true,
	linkcolor=RoyalBlue,
	citecolor=gray,
	urlcolor=Cyan,
	pdfauthor={eyemask57},
}
\sisetup{separate-uncertainty,per-mode=symbol,detect-all,range-phrase=--}

\ExecuteBibliographyOptions{
	abbreviate=true,
	maxnames=8,
	minnames=3,
	hyperref=auto,
	backref=true,
	abbreviate=true,
	date=year,
	arxiv=abs,
	isbn=true,
	url=true,
	doi=true
	}

\setcounter{tocdepth}{3}

%\setmainfont{Times-New-Roman}[BoldFont=Times-New-Roman]
%\setsansfont{Arial}

\begin{comment}
\setmainjfont[
YokoFeatures       = {JFM=jlreq},
TateFeatures       = {JFM=jlreqv},
BoldFont           = Yu-Gothic,
BoldFeatures       = {FakeBold=2},
ItalicFont         = Yu-Mincho,
ItalicFeatures     = {FakeSlant=0.33},
BoldItalicFont     = Yu-Gothic,
BoldItalicFeatures = {FakeBold=2, FakeSlant=0.33}
]{Yu-Mincho}
\setsansjfont[
YokoFeatures       = {JFM=jlreq},
TateFeatures       = {JFM=jlreqv},
BoldFont           = Yu-Gothic,
BoldFeatures       = {FakeBold=2},
ItalicFont         = Yu-Gothic,
ItalicFeatures     = {FakeSlant=0.33},
BoldItalicFont     = Yu-Gothic,
BoldItalicFeatures = {FakeBold=2, FakeSlant=0.33}
]{Yu-Gothic}
\end{comment}

\DeclareEmphSequence{\gtfamily\sffamily}

\def\qed{\rightline\square}
\def\qedhere{\quad\square}

\newcommand{\Aut}{\operatorname{Aut}}
\newcommand{\id}{\operatorname{id}}
\newcommand{\spanL}{\operatorname{span}}
\newcommand{\Ker}{\operatorname{Ker}}
\newcommand{\lcm}{\operatorname{lcm}}
\newcommand{\notmid}{\mathrel{\not|}}

\lstnewenvironment{mylisting}[1][]
    {\lstset{
        frame=single,
        basicstyle=\ttfamily,
        numbers=left,
        numbersep=10pt,
        tabsize=2,
        extendedchars=true,
        xleftmargin=17pt,
        framexleftmargin=17pt,
        #1
    }
}{}

\lstset{%
  basicstyle={\small},%
  identifierstyle={\small},%
  commentstyle={\small\itshape},%
  keywordstyle={\small\bfseries},%
  ndkeywordstyle={\small},%
  stringstyle={\small\ttfamily},
  frame={tb},
  breaklines=true,
  columns=[l]{fullflexible},%
  numbers=left,%
  xrightmargin=0zw,%
  xleftmargin=3zw,%
  numberstyle={\scriptsize},%
  stepnumber=1,
  numbersep=1zw,%
  lineskip=-0.5ex%
}

\definecolor{burgundy}{rgb}{0.5, 0.0, 0.13}
\tcbset{mytheo/.style={fonttitle=\gtfamily\sffamily\bfseries\upshape,
enhanced,
breakable,
colframe=burgundy,
colback=burgundy!2!white,
colbacktitle=burgundy,
boxrule=0pt,
borderline south={2pt}{-2pt}{burgundy},
left*=1\zw,
right*=1\zw,
theorem style=standard,
sharp corners,
before skip=8pt,
after skip=10pt,
before upper={\setlength{\parindent}{1\zw}},
before lower={\setlength{\parindent}{1\zw}}
}}

\newtcbtheorem[number within=section]{thm}{Theorem}%定理
{mytheo}{tm}
\newcommand{\thmref}[1]{{\bfseries\sffamily Theorem~\ref{tm:#1}}}

\newtcbtheorem[use counter from=thm]{prop}{Proposition}%命題
{mytheo}{pro}
\newcommand{\propref}[1]{{\bfseries\sffamily Proposition~\ref{pro:#1}}}

\newtcbtheorem[use counter from=thm]{cor}{Corollary}%系
{mytheo}{co}
\newcommand{\corref}[1]{{\bfseries\sffamily Corollary~\ref{co:#1}}}

\newtcbtheorem[use counter from=thm]{dfn}{Definition}%定義
{mytheo,
colframe=blue!50!black,colback=blue!50!black!2!white,colbacktitle=blue!50!black,borderline south={2pt}{-2pt}{blue!50!black},}{de}
\newcommand{\dfnref}[1]{{\bfseries\sffamily Definition~\ref{de:#1}}}

\newtcbtheorem[use counter from=thm]{ax}{Axiom}%公理
{mytheo,
colframe=blue!50!black,colback=blue!50!black!2!white,colbacktitle=blue!50!black,borderline south={2pt}{-2pt}{blue!50!black},}{axi}
\newcommand{\axref}[1]{{\bfseries\sffamily Axiom~\ref{axi:#1}}}

\newtcbtheorem[use counter from=thm]{lem}{Lemma}%補題
{mytheo,
colframe=green!50!black,colback=green!50!black!2!white,colbacktitle=green!50!black,borderline south={2pt}{-2pt}{green!50!black},}{le}
\newcommand{\lemref}[1]{{\bfseries\sffamily Lemma~\ref{le:#1}}}

%\newcounter{myexample}
%[usecounter=myexample,numberformat=\Alph]
%[number within=section]
\definecolor{charcoal}{rgb}{0.21, 0.27, 0.31}
\newtcbtheorem[use counter from=thm]{ex}{Example}%例
{mytheo,
colframe=charcoal,colback=charcoal!2!white,colbacktitle=charcoal,borderline south={2pt}{-2pt}{charcoal},}{exa}
\newcommand{\exref}[1]{{\bfseries\sffamily Example~\ref{exa:#1}}}

\newtcbtheorem[use counter from=thm]{qu}{Question}%例
{mytheo,
colframe=charcoal,colback=charcoal!2!white,colbacktitle=charcoal,borderline south={2pt}{-2pt}{charcoal},}{que}
\newcommand{\quref}[1]{{\bfseries\sffamily Question~\ref{que:#1}}}

%----------------------------------------------------------------------

\newtcbtheorem[use counter from=thm]{con}{Conjection}%予想
{mytheo,
colframe=magenta,colback=magenta!2!white,colbacktitle=magenta,borderline south={2pt}{-2pt}{charcoal},}{cn}
\newcommand{\conref}[1]{{\bfseries\sffamily Conjection~\ref{cn:#1}}}

\newtcbtheorem[use counter from=thm]{thy}{Theory}%理論
{mytheo,
colframe=yellow!50!black,colback=yellow!50!black!2!white,colbacktitle=yellow!50!black,borderline south={2pt}{-2pt}{yellow!50!black},}{ty}
\newcommand{\thyref}[1]{{\bfseries\sffamily Theory~\ref{ty:#1}}}

\newtcbtheorem[use counter from=thm]{pr}{Principle}%性質,原理
{mytheo,
colframe=yellow!50!black,colback=yellow!50!black!2!white,colbacktitle=yellow!50!black,borderline south={2pt}{-2pt}{yellow!50!black},}{pri}
\newcommand{\prref}[1]{{\bfseries\sffamily Principle~\ref{pri:#1}}}

\newtcbtheorem[use counter from=thm]{req}{Requestment}%要請
{mytheo,
colframe=yellow!50!black,colback=yellow!50!black!2!white,colbacktitle=yellow!50!black,borderline south={2pt}{-2pt}{yellow!50!black},}{rq}
\newcommand{\reqref}[1]{{\bfseries\sffamily Requestment~\ref{rq:#1}}}

\newtcbtheorem[use counter from=thm]{res}{Resalt}%結果
{mytheo,
colframe=yellow!50!black,colback=yellow!50!black!2!white,colbacktitle=yellow!50!black,borderline south={2pt}{-2pt}{yellow!50!black},}{rs}
\newcommand{\resref}[1]{{\bfseries\sffamily Resalt~\ref{rs:#1}}}

\newtcbtheorem[use counter from=thm]{law}{Law}%法則
{mytheo,
colframe=yellow!50!black,colback=yellow!50!black!2!white,colbacktitle=yellow!50!black,borderline south={2pt}{-2pt}{yellow!50!black},}{la}
\newcommand{\lawref}[1]{{\bfseries\sffamily Law~\ref{la:#1}}}

%\newtcbtheorem[use counter from=thm]{}{}%
%{mytheo}{}
%\newcommand{\_ref}[1]{{\bfseries\sffamily ~\ref{:#1}}}

\newtcolorbox{proof}{
% invisible,
% ignore,
  breakable,
  empty,
  coltitle=black,
  title=\itshape Proof.,
  left=0pt,
  right=0pt,
  top=0pt,
  bottom=0pt,
}

\newtcolorbox{answer}{
% invisible,
% ignore,
  breakable,
  empty,
  coltitle=black,
  title=\itshape Answer.,
  left=0pt,
  right=0pt,
  top=0pt,
  bottom=0pt,
}

\newtcolorbox{remark}[1][]{
% invisible,
% ignore,
  enhanced,
  before skip=2mm,after skip=3mm,fontupper=\gtfamily\sffamily,
  boxrule=0.4pt,left=5mm,right=2mm,top=1mm,bottom=1mm,
  colback=yellow!50,
  colframe=yellow!20!black,
  sharp corners,rounded corners=southeast,arc is angular,arc=3mm,
  underlay={%
    \path[fill=tcbcolback!80!black] ([yshift=3mm]interior.south east)--++(-0.4,-0.1)--++(0.1,-0.2);
    \path[draw=tcbcolframe,shorten <=-0.05mm,shorten >=-0.05mm] ([yshift=3mm]interior.south east)--++(-0.4,-0.1)--++(0.1,-0.2);
    \path[fill=yellow!50!black,draw=none] (interior.south west) rectangle node[white]{\Huge\bfseries !} ([xshift=4mm]interior.north west);
    },
  drop fuzzy shadow,#1}

\newcounter{reidaibangou} %%カウンタの定義
\newtcolorbox{reidai}[1][]{breakable,enhanced,boxrule=0.5mm,
	top=2pt,left=44pt,right=4pt,bottom=2pt,arc=0mm,
	colframe=blue!30!gray,
	boxrule=1pt,
	underlay={
		\node[inner sep=1pt,blue!50!black,fill=blue!10!white]at ([xshift=22pt,yshift=-9pt]interior.north west) {\stepcounter{reidaibangou}\bfseries\gtfamily Exexercise\thereidaibangou};},
	segmentation code={%
		\draw[dashed] (segmentation.west)--(segmentation.east);
		\node[inner sep=1pt,blue!50!black,fill=blue!10!white] at ([xshift=22pt,yshift=-8pt]segmentation.south west) {\bfseries\gtfamily Ans.};},
	before upper={\setlength{\parindent}{1zw}},
	before lower={\setlength{\parindent}{1zw}},
}
%%%%%ここまでがreidai環境の定義。例えば本文中に以下のように記述してみよう。