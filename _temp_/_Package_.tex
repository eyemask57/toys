\documentclass[
	head_space=15mm,
	food_space=25mm,
	gutter=20mm,
	fore-adge=20mm,
	a4paper,
%	titlepage,
	fontsize=10.5bp]{jlreq}
	%基本はarticle
	%オプションに記入していく: book, report
%\documentclass{beamer}
	%スライド用
	%LuaLaTeXでコンパイルする前提

\usepackage{luatexja-ruby,luacode,luacolor}
	%ルビを振れる,luaコードを利用できる,luaLaTeX用のcolorパッケージ

%\usepackage[T1]{fontenc}
%\usepackage{lmodern}

\usepackage[no-math]{fontspec}
	%必須.Unicordを使うやつ
\usepackage{luatexja-otf}
	%Unicordを入力できるようにする
\usepackage{luatexja-fontspec}
%\setmainjfont{UDDigiKyokashoNP-R}[BoldFont=UDDigiKyokashoNP-B]
%\setmainfont{UDDigiKyokashoNP-R}[BoldFont=UDDigiKyokashoNP-B]
	%本文中のフォント設定

%\usepackage[top=15mm, hmargin=20mm, bottom=25mm]{geometry}
	%ページの余白を指定することができる
	%jlreqより休止中

\usepackage{bxpapersize}
	%サイズ指定するだけ
\usepackage{comment}
	%長文をコメントアウトする
%comment環境を用いる
	%インデントしてはいけない

%\usepackage{lipsum}
	%英文用だみーてきすと

%\usepackage{datestamp}
	%なんどコンパイルしても消えないタイムスタンプを残せる

%\usepackage{tocloft}
	%目次についていじれる


%\usepackage{appendix}
	%補遺の制御を簡単にする

%\usepackage{abstract}
	%abstractをいじれる
	%- オプションなし
	%- number
	%- runin
	%- original
	%- style
	%- addtotoc

%\usepackage{pdfpages}
	%pdfをそのままページとして挿入可能
%\usepackage{Bxpdfver}
	%まれに必要になるらしいやつ.休止中

\usepackage{amsmath,amssymb,amsfonts,mathtools,empheq,autobreak,breqn,esint,latexsym}
	%必須.数学の記号・ギリシャ文字などの拡張

%\usepackage{amsthm}
	%定理環境の拡張.現在はtcolorbox使用により停止中

\usepackage{bm,bbm}
	%数式内で太字を楽に出力.\bm{}で出力
\usepackage{euscript,mathrsfs}
	%数式内の筆記体(カリグラフィー),花文字(スクリプトフォント)を追加.\mathscr

\usepackage{upgreek}
	%立体ギリシャ文字を追加
\usepackage[bbgreekl]{mathbbol}
\DeclareSymbolFontAlphabet{\mathbb}{AMSb}
\DeclareSymbolFontAlphabet{\mathbbl}{bbold}
\DeclareMathSymbol{\bbepsilon}{\mathord}{bbold}{"0F}
	%黒板文字の追加.
	%定義がみすってるらしいので\bbepsilonで再定義している.
	%amsfontを大文字で使いたいので\mathbblを別途定義

%\usepackage[geometry,misc]{ifsym}
%\usepackage{textcomp}
	%記号の追加多数

\usepackage{physics2}
	%物理でよく用いる記号の拡張
	%physics packageは問題点が多すぎるため使用は非推奨
%\usepackage{askmaps}
	%カルノー図を描く
%\usepackage{mhchem}
	%化学式・化学反応式.\ce{}で出力
\usepackage{chemmacros}
\chemsetup{ modules = all }

%\usepackage{mleftright}
	%かっこのさゆうの空白を調節した\mleft,\mrightをついか
\usepackage{diffcoeff}
	%微分記法を提供してくれる
\usepackage{nicematrix}
	%自由度の高い行列やNiceTabular環境の提供
%\usepackage{esvect}
	%矢線ベクトル表記のバリエーションを提供
	%下付き文字のためのコマンドを提供
%\usepackage{tensor}
	%テンソル表記のためのコマンドを提供
\usepackage{braket}
	%Diracのbra-ket記法のためのコマンドを提供
	%集合論における内包的記法のためのコマンドを提供
%\usepackage{centernot}
	%Feynmanのスラッシュ記法のためのコマンドを提供
\usepackage{accents}
	%数学記号に複数のアクセントをつけられるようにする
%\usepackage{witharrows}
	%数式環境で式変形の説明用の矢印をかける
\usepackage{cascade}
	%推論規則を書ける


\usepackage[thicklines]{cancel}
	%消す項につける斜線の出力
\usepackage{siunitx}
	%SI単位の出力.単位のみは\si{},数字を入れるときは\SI{}{}
%\usepackage{polyglossia}
	%多言語対応のためのパッケージ
\usepackage{enumitem}
	%箇条書きのレパートリーを増やす

\usepackage{graphicx}
	%画像の挿入・拡大・回転を行う
\usepackage{float}
	%\figureの[H]

\usepackage{array}
	%tabular環境の線などの出力を微妙に改善
	%標準は\hline
%\usepackage{diagbox}
	%セル内に斜線を引く
%\usepackage{dcolumn}
	%小数点で整列できる
%\usepackage{longtable}
	%ページをまたぐ表が作れる
%\usepackage{truthtable}
	%真理値表の出力
%\usepackage{booktabs}
	%横三本のみの線による表を作成.縦は出力出来ない
	%tabular環境で使用
	%\toptabs \midtabs \bottomtabsで線を出力
%\usepackage{tabularx}
	%表の横幅を指定できるようになる

\usepackage{tikz}
\usetikzlibrary{intersections,calc,arrows.meta,cd}
	%複雑な図が書ける
%\usepackage{tikz-feynman}
	%Feynmanダイアグラムをかける
\usepackage{tikz-cd}
	%可換図式を書ける
%\usepackage{gnuplot-lua-tikz}
	%グラフを書いてくれるらしい.なにこれ
\usepackage[dvipsnames]{xcolor}
\usepackage{tcolorbox}
	%いろんな枠組みが書ける
\tcbuselibrary{most}
	%tcolorboxのトモダチ

\usepackage{framed}
	%簡単な枠組みが書ける

%\usepackage{listings,jvlisting}
	%ソースコードが綺麗に書けるように設定できる

\usepackage{hyperref}
	%ハイパーリンクや参照などいろんなことができる
%\usepackage {intopdf}
	%他のPDFファイルをハイパーリンクを使って読み込む

%\usepackage{csquotes}
\usepackage{biblatex}
\renewcommand{\bibnamedash}{%
\hskip.2em \leavevmode\rule[.5ex]{2.5em}{.3pt}\hskip0.4em}
	%bibtexの上位互換.UTF-8に対応したbiberを用いる.

	%↓パッケージをつくれるらしい.
	%↓そうでなくても\newcommandではできないコマンドが作れる.
%\usepackage{xparse}
	%オプション因数付きコマンドを定義できる
%\usepackage{ifthen}
	%条件分岐と反復処理
%\usepackage{xkeyval}
	%key-value 形式によるコマンドの定義が可能
%\usepackage{etoolbox}
	%なんかいろいろできる
%\usepackage{calc}
	%レイアウトの演算ができる

%\usepackage{bxtexlogo}
	%TeXのロゴ
%\usepackage{manfnt}
	%危険な曲がり角が書ける
%\usepackage{scsnowman}
	%雪だるまが書ける
%\usepackage{coffee stains}
	%コーヒーのシミが書ける
%\usepackage{halloweenmath}
	%ハロウィンな数式を書ける
%\usepackage{bearwear}
	%テディベアのきせかえができる
%\usepackage {tikzducks}
%\usetikzlibrary{ducks}
	%アヒルで遊べる
%\usepackage{randomwalk}
	%ランダムウォークできる